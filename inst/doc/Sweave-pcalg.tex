% -*- mode: noweb; noweb-default-code-mode: R-mode; -*-
\documentclass[a4paper]{article}

\title{\texttt{pcalg}: A brief overview}
\author{Markus Kalisch}
%\VignetteIndexEntry{pcalg}

\usepackage{a4wide}

\usepackage{/usr/local64.sfs/app/R/R-2.3.1-ATLAS-inst/share/texmf/Sweave}
\begin{document}

\maketitle
\section{General Introduction}
In this document, we will give a short overview of the R-package
\texttt{pcalg}. The package can be used to analyze dependencies among
random variables. The main functions are
\begin{itemize}
\item \texttt{pcAlgo}: (also see \texttt{?pcAlgo}) Use the PC-Algorithm to
  estimate the skeleton of a 
  DAG. A DAG is a Directed Acyclic Graph and the skeleton of a DAG is the
  DAG without the arrowheads. In practice, the skeleton will only be a good
  estimate, if you have 
  way more samples than nodes. If not, you will only get the edges, that
  show the most obvious ``direct'' dependencies (``direct'' means: The
  dependence didn't go away, even when conditioning on any subset of the
  remaining variables). Using the plot option \texttt{zvalue.lwd} you can
  further qualify the edges by choosing their line width according to their
  reliability in the statistical tests. I.e., thick lines show reliable
  dependencies. The value of this function is an object of S4 class
  ``pcAlgo'' (see \texttt{?pcAlgo-class}) on which you can use
  \texttt{plot}, \texttt{summary} and \texttt{show}.
\item \texttt{corGraph}: (see also \texttt{?corGraph}) Each pair of
  variables is tested for zero 
  correlation on a given significance level. If a test is rejected, the
  corresponding edge in the graph is kept. I.e., the graph shows an edge
  between correlated nodes.
\end{itemize}
Furthermore, there are some functions that generate random data and perform
statistical tests.

\section{Some examples}
In this section, we will show some simple examples with the most important
functions.
First, we generate a data set that will be used later on:
\begin{Schunk}
\begin{Sinput}
> library(pcalg)